\textbf{Glosario}

\begin{longtable}{ | p{2.5cm} | p{4.5cm} |  }
	\caption{Glosario del manual de instalación}
	\label{table:glosariomanuinsta}
	\\	\hline
		 \textbf{Término} &\textbf{Descripción}
        \\ \hline
        \textbf{Aplicación de escritorio}&Es aquella que se encuentra instalada en 
        el disco duro de la computadora y no requiere de conexión a internet para 
        que el usuario pueda hacer uso de sus funciones.
        \\ \hline
        \textbf{Consola} &Es el programa informático que provee una interfaz de 
        usuario para acceder a los servicios del sistema operativo.
        \\ \hline
        \textbf{Lenguaje de programación}&Es un lenguaje formal diseñado para 
        realizar procesos que pueden ser llevados a cabo por máquinas como las 
        computadoras. Estos pueden usarse para crear programas que controlen el 
        comportamiento físico y lógico de una máquina, para expresar algoritmos 
        con precisión, o como modo de comunicación humano.
        \\ \hline
        \textbf{Framework}&Es un conjunto estandarizado de conceptos, prácticas y 
        criterios para enfocar un tipo de problemática particular, que sirve como 
        referencia para enfrentar y resolver nuevos problemas de índole similar.
        \\ \hline
\end{longtable}
%Falta cluster, open source, terminal/consola, ejecutable, interfaz, servidor