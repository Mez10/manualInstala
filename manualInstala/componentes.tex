\begin{enumerate}
    \item \textbf{Procesamiento de vídeo:}
    Es el componente que a partir del flujo del video obtiene los frames
    para ello se toma en cuenta la codificación.
    Los frames obtenidos del video se eligen los que cumplen 
    con tres marcas de un nadador.
    \item \textbf{Segmentación de la imagen por color:}
    Se detectan las tres marcas en el brazo del nadador 
    para evaluar su brazada:
    \begin{itemize}
        \item Color verde la del hombro.
        \item Color amarillo la del codo.
        \item Color rojo en la muñeca.
    \end{itemize}
    En la patada se contará con la detección de tres 
    marcas que están colocadas en el siguiente orden:
\begin{itemize}
   \item Color verde en la parte baja de la rodilla.
   \item Color amarillo en el tobillo.
   \item Color rojo recubriendo parte del empeine, recubriendo lateralmente 
   y concluyendo en la parte de la planta del pie, justo por debajo del empeine.
\end{itemize}
    Los colores de cada marca son convertidos en una máscara binaria.
\\*
    \item \textbf{Obtención del cluster (identificación de la marca):}
    En esta etapa del proceso se identifica la marca de color previamente 
    segmentada, procurando minimizar el ruido resultante de la binarización de 
    la imagen.\\
    Se estima que, de acuerdo a la calidad de la segmentación de color, la
    identificación de la marca se pueda hacer mediante agrupación de píxeles 
    semejantes mediante búsqueda en profundidad.
    Finalmente, el cluster resultante es adaptado a una forma regular y se obtiene el
    centroide de dicha figura.
\\*
    \item \textbf{Cálculo de vectores:}
    Por medio del centroide de cada marca, se unen matemáticamente las tres marcas de 
    color: las marcas roja, verde y amarilla formaran un triángulo. Con ello, podrán
    ser calculados dos vectores para poder conformar la fórmula del arco coseno.
    Se debe tomar como origen un centroide para calcular un ángulo:

\begin{longtable}{ | p{5.5cm} | p{5.5cm} | }
	\caption{Marca origen respecto del ángulo a calcular}
	\label{table:marcaorigen}
	\\	\hline
		 \textbf{Marca origen} &\textbf{Ángulo}
		\\ \hline
        \textbf{Verde}&Ángulo de muñeca.
        \\ \hline
        \textbf{Amarilla}&Ángulo hombro.
        \\ \hline
        \textbf{Roja}&Ángulo de codo.
        \\ \hline
\end{longtable}

    \item \textbf{Reconocimiento de patrones:}
    En esta fase, se evaluará que los ángulos obtenidos estén dentro de un rango 
    definido con base en literatura basada en la biomecánica de la natación, 
    así como en los valores obtenidos mediante el análisis de grabaciones 
    realizadas a nadadores de alto rendimiento. Esto, con la finalidad de adaptar 
    la evaluación de patrones a la envergadura y complexión de nadadores mexicanos.
\\*
    \item \textbf{Visualización de resultados:}
\\*
    Elaboración de un reporte de resultados dirigido al entrenador. Dicho reporte 
    contendrá datos tanto de la brazada como de la patada del nadador.
    Contendrá datos tanto del nadador como del entrenador.
\end{enumerate}