\begin{itemize}
    \item \textbf{Instalación y configuración del software base}
    \\
    \textbf{Python}
    \\
    Es un lenguaje de programación interpretado, multiparadigma que
    soporta programación orientada a objetos, programación imperativa
    y programación funcional.
    \\
    \item \textbf{Localización}
    \\
    Se localiza en el Disco C (normalmente, la partición principal 
    del disco duro en Windows), en la subcarpeta Archivos de programa.
    \\
    \item \textbf{Procedimiento de Instalación}
    \\
    \begin{itemize}
        \item Entrar al sitio oficial de Python.
        \item Dar click en el que corresponde al sistema operativo Windows.
        \item Descargar la version 3.x con arquitectura x64 o x86 según 
        corresponda la versión de Windows instalada en su PC.
        \item Ejecute el instalador de Python dando doble clic sobre el
        archivo ejecutable.
        \item Dar clic en \textbf{siguiente} hasta llegar al botón \textbf{finalizar}.
        \item Para verificar que está instalado Python, favor de seguir estos pasos:
        \begin{enumerate}
            \item Dar clic en el botón \textbf{inicio}.
            \item Clic en la opción Ejecutar y escribir \textbf{cmd}.
            \item Pulsar Enter
            \item Se abrirá la consola de Windows
            \item Buscar la carpeta de C:\textbackslash{python3.x} y debe mostrar unos 
            cocodrilos y podras dar enter y salir de la consola de python por medio 
            del comando exit().
        \end{enumerate}
    \end{itemize}

        \item \textbf{Procedimiento de configuración}
        \\
        \begin{enumerate}
            \item Modificar las variables de entorno
            \item Modificar el PATH. Para ello, dar doble clic en editar, con ello se 
            situará hasta el cursor hasta el final de la línea. Allí es donde se deberá 
            agregar la ruta \textbf{C:\textbackslash{python3.x}}
            \item Fin de los pasos de instalación.
        \end{enumerate}
\end{itemize}  

\textbf{Open CV}
\\
Librería open source para procesamiento de imágenes. Principalmente utilizada para
computación visual cuenta con estructuras básicas de datos como optimización de matrices
y procesamiento de video.
\textbf{Localización}
\\
Se localiza en el disco C y en los archivos de programa.
\\
\textbf{Procedimiento de Instalación}
\\
\begin{enumerate}
    \item Bajar el Open CV 3 package del sitio oficial. 
    \item Dar click en el que corresponde al sistema operativo Windows.
    \item Instalar en c:/opencv.
    \item Instalar mini conda 
    \item Entrar a la consola de mini conda. 
    \item En la terminal de mini conda escribir: \textbf{conda install opencv3}
    \item Verificar que fue instalado accediendo a python en la terminal, luego, escribir
    \textbf{import cv2} y pulsar Enter, hecho esto, deben aparecer los cocodrilos de python
    en la terminal.
    \item No debe aparecer ningun mensaje de error.
    \item Fin de la instalación.
\end{enumerate}
\textbf{Procedimiento de configuración de OpenCV 3}
\\
Se verifica que este instalado en la siguiente ruta:
\begin{itemize}
    \item C:\textbackslash{Users} \textbackslash{Anaconda}
    \item C:\textbackslash{Users} \textbackslash{Anaconda} \textbackslash{Scripts}
\end{itemize}
De acuerdo con la arquitectura de su PC (x64 o x86), debería ver el directorio \textbf{opencv}
en la ruta que corresponda según la siguiente tabla:
\begin{longtable}{ | p{5.5cm} | p{5.5cm} | p{5.5cm} | }
	\caption{Directorios de OpenCV por arquitectura}
	\label{table:opencvarquitectura}
	\\	\hline
		 \textbf{Arquitectura} &\textbf{Directorio}&\textbf{Ruta}
		\\ \hline
        \textbf{32 bits}&OPENCV\_DIR&C:\textbackslash{opencv} \textbackslash{build} \textbackslash{x86} \textbackslash{vc12}
        \\ \hline
        \textbf{64 bits}&OPENCV\_DIR&C:\textbackslash{opencv} \textbackslash{build} \textbackslash{x64} \textbackslash{vc12}
        \\ \hline
\end{longtable}
\textbf{Parámetros a configurar}
\begin{itemize}
    \item Ninguno.
\end{itemize}   
 
\textbf{MySql}
\\
Es un sistema de gestión de bases de datos relacional desarrollado bajo licencia dual
GPL/Licencia comercial por Oracle Corporation y está considerada como la base datos
open source más popular del mundo.
\\
\textbf{Localización}
\\
Por defecto, se localiza en el Disco C (normalmente, la partición principal del 
disco duro en Windows), en la subcarpeta Archivos de programa.
\textbf{Procedimiento de Instalación}
\\ 
\begin{itemize}
    \item Entrar al sitio oficial de MySql 
    \item Dar click en el que corresponde al sistema operativo Windows.
    \item Descargar la versión mysql-installer-community 5.5 
    \item Abrir el ejecutable.
    \item Hacer doble clic en setup.exe para iniciar el proceso de configuración.
    \item Hacer clic en \textbf{personalizar}.
    \item Hacer clic en \textbf{cambiar}.
    \item En la siguiente ventana en el campo de texto titulado Folder Name, 
    se cambiará la ruta por \textbf{C:\textbackslash{ServerMySQL}} y dar clic en aceptar.
    \item En la siguiente ventana hacer clic en \textbf{siguiente} y MySQL estará 
    listo para instalarse.
    \item Se sugiere saltar el registro y dar \textbf{siguiente}.
    \item La instalación estará lista cuando aparezca la ventana MySQL Sing-Up.
\end{itemize}
\textbf{Procedimiento de configuración}
\begin{itemize}
   \item Para iniciar la configuración dejar marcado MySQL 
   Server Now y dar clic en \textbf{finalizar}.
   \item Dar clic en Server Machine dejar el puerto por defecto (3306).
   \item Elegir la configuración estándar y dar clic en siguiente.
   \item Verificar que están marcados:
   \begin{itemize}
       \item Install As Windows Service.
       \item Launch the MySQL Server Automatically.
   \end{itemize}
    \item Dar clic en siguiente.
    \item Crear una contraseña de administrador como se solicita en la ventana.
    Verificar que root este marcado como \textbf{enable root access from remote 
    machine} y dar clic en siguiente.
    \item Dar clic en ejecutar para iniciar el servidor de MySQL y dar clic en finalizar.
    \item Entrar a una consola de MySQL y escribir el siguiente comando: mysql -u root -p
    y presione Enter; acto seguido, se le pedirá que ingrese la contraseña de administrador
    que configuró en pasos anteriores.
    \item No debe salir ningun error.
    \item Se termino el proceso de configuración.
\end{itemize}
\textbf{SQL Alchemy}
\\
Es un Object Relational Mapper, es presentado y descrito completamente para python. La 
persistencia es automática.
\textbf{Localización}
\\
Por defecto, se localiza en el Disco C (normalmente, la partición principal del 
disco duro en Windows), en la subcarpeta Archivos de programa.
\\
\textbf{Procedimiento de Instalación}
Los pasos son los siguientes:
\begin{itemize}
    \item Ir al sitio oficial de SQL Alchemy y descargar el ejecutable según la 
    de su PC, es decir, 32 bits o 64 bits.
    \item Entrar a la consola y entrar a carpeta \textbf{C:\textbackslash{Python3.x}\textbackslash{python.exe}}
    \textbackslash{setup.py} install
    \item Instalación finalizada
\end{itemize}
\textbf{Procedimiento de configuración}
\begin{itemize}
    \item Ninguno
\end{itemize}
\textbf{Pyramid}
\\
Es un framework para construir aplicaciones web desarrolladas con python.
\\
\textbf{Localización}
\\
Por defecto, se localiza en el Disco C (normalmente, la partición principal del 
disco duro en Windows), en la subcarpeta Archivos de programa.
\\
\textbf{Procedimiento de Instalación}
\\
Los pasos son los siguientes:
\begin{itemize}
    \item Ir al sitio oficial de Pyramid y descargar el ejecutable según la 
    de su PC, es decir, 32 bits o 64 bits.
    \item Entrar a la consola de Windows e instalar con los siguientes comandos:
    \begin{enumerate}
        \item pip install pyramid==1.9.1
    \end{enumerate}
    \item La instalación habrá finalizado cuando no se envíe ningun mensaje de error.
\end{itemize}

\textbf{Procedimiento de configuración}
\begin{itemize}
    \item Ninguno.
\end{itemize}