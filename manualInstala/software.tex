\begin{itemize}
\item \textbf{Matriz de certificación:}
Para el sistema Windows se instalarán las siguientes
librerías. Se recomienda instalarlas desde la 
aplicación \textbf{mini conda}.
\begin{longtable}{ | p{3.0cm} | p{5.5cm} | p{6.0cm} | }
	\caption{Instalación de librerías}
	\label{table:librerias}
	\\	\hline
		 \textbf{Librería} &\textbf{Comando}&\textbf{Uso}
		\\ \hline
        \textbf{Open CV}&conda install -c menpo opencv3&Procesamiento de vídeo como el 
        análisis de las imágenes.
        \\ \hline
        \textbf{Numpy}&conda install numpy&Para el procesamiento de matrices como 
        operaciones matemáticas.
        \\ \hline
        \textbf{Pillow}&conda install pillow&Brinda al intérprete de python la 
        facultad de procesar imágenes.
        \\ \hline
        \textbf{PyGame}&conda install -c cogsci pygame&Conjunto de módulos del 
        lenguaje Python que permiten la creación de videojuegos en dos dimensiones de 
        una manera sencilla. Está orientado al manejo de sprites.
        \\ \hline
        \textbf{Sympy}&conda install sympy&Librería para computación geográfica.
        \\ \hline
        \textbf{Matplotlib}&conda install matplotlib&Creación de graficas  y plots.
         \\ \hline
\end{longtable}
\item \textbf{Restriciones técnicas del sistema}
\begin{longtable}{ | p{6.0cm} | p{5.5cm} | p{5.5cm} | }
	\caption{Restricciones técnicas}
	\label{table:restriccionestec}
	\\	\hline
		 \textbf{Elemento} &\textbf{Descripción}
		\\ \hline
        \textbf{Sistema operativo}&Windows 7, 8, 8.1 o 10
        \\ \hline
        \textbf{Service pack}&SP1 en adelante
        \\ \hline
        \textbf{Sistema gestor de base de datos}&MySQL
        \\ \hline
        \textbf{Intérprete}&Python Shell
        \\ \hline
        \textbf{Tipo de software}&Escritorio o Stand alone
        \\ \hline
\end{longtable}
\end{itemize}