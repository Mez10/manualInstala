%Crea la instancia del documento. Le indica a Latex como "imprimir" el documento
%que escribes. Nota que al final, junto a "11pt", existe una palabra que dice
%"Final". en ese lugar puedes tener dosvalores:
%    final - Indica que la version a compilar (e imprimir) es el documento final
%            Analogamente a programacion, digamos que es "produccion"
%    draft - Indica que el documento a compilar (e imprimir) esta en edicin.
%            Esto provoca que compile mas rapido pero, como resultado, 
%            los hipervinculos, y las imagenes, no seran imprimidas en el documento.
%            Es bastante util cuando tienes un documento con mas de 15 imagenes, ya que
%            las imagenes alentan bastante el proceso de compilado.
%            Mientras no tengas imagenes, o no quieras ver las imagenes, deja el
%            valor de "draft".
\documentclass[oneside,11pt,final]{book}
    %Margenes especificados por la CATT
    \usepackage[letterpaper, lmargin=3cm, rmargin=2cm, tmargin=2cm, bmargin= 2cm]{geometry}
    \usepackage[spanish, activeacute]{babel} %Definir idioma spanish
    \usepackage[utf8]{inputenc} %Para usar acentos y otros caracteres UTF-8
    \usepackage[nottoc]{tocbibind} %Para que la seccion de referencias aparezca
                                   %como entrada en el indice
    \usepackage{enumerate}
    %\usepackage[hidelinks]{hyperref} %Para que notas, referencias e indice se
    %hipervinculicen automaticamente
    \usepackage{graphicx}
    \usepackage{wrapfig}
    \usepackage{microtype}
    \usepackage{vwcol}
    \usepackage{longtable} %Para tablas gigantes
    %Definir ruta de graficos
    \graphicspath{ {imagenes/general/}{imagenes/diagramas/} }
    %No usar hasta que tengas glosario.
    %\usepackage[xindy,toc]{glossaries} %Para incluir entradas de glosario
    %\makeglossaries
    %Cargar el archivo glosario.tex
    %\loadglsentries{glosario}
    \usepackage{amssymb}
    \usepackage[usenames,dvipsnames,table]{xcolor}
    \usepackage{amsmath}
    \usepackage{color}
    \usepackage{tikz}
    \usepackage[final]{pdfpages}
    \usepackage{wallpaper,lipsum}
    \usepackage{colortbl}
    \usepackage{txfonts}
    \usepackage{mathdots}
    \usepackage{subfigure}
    \usepackage{sepacrolTheme}
    \usepackage{sepacrolAnalisis}
    \usepackage{sepacrol}
    \usepackage{epsfig}
    \usepackage{multirow}
    
    \usepackage[classicReIm]{kpfonts}
    %Definir el titulo del TT
    \title{\textbf{Sistema de evaluación de la técnica del estilo crol en natación utilizando patrones de distancia (SEPACROL)}}
    %Definir el nombre de los integrantes.
    %( los "\\" son para insertar un salto de linea)
    \author{Hoyos Estrada Edgar Omar\\
            López Zaragoza Joscelyn Meztli}
    
    \begin{document}
    \frontmatter
    
\setlength{\unitlength}{1 cm} %Especificar unidad de trabajo
\thispagestyle{empty}
%\addtolength{\headsep}{-6cm}
\begin{picture}(18,4)
\put(-2,-0.8){\includegraphics[width=2.5cm,height=3.5cm]{logo_ipn}}
\put(-2.3,-18.5){\includegraphics[width=3cm,height=2.2cm]{logo_escom}}
\put(-1, -16){\tikz \fill [black] (0.6,0.1) rectangle (0.2,15.0);}
\end{picture}
\begin{center}
{\textbf{{\Huge Instituto Politécnico Nacional}}\\[0.5cm]
	{\LARGE \textbf{Escuela Superior de Cómputo}}}\\[1.25cm]
{\LARGE \textbf{ESCOM}}\\[1.25cm]
{\LARGE \emph{Manual de instalación}}\\[1.5cm]
{\Large \textbf{``Sistema de evaluación de la técnica del estilo crol en natación utilizando patrones de distancia (SEPACROL)''}}\\[0.5cm]
{\large 2016 - B006}\\[1.25cm]
{\large \emph{Presentan}}\\[0.5cm]
{\large \textbf{Hoyos Estrada Edgar Omar}}\\[0.15cm]
{\large \textbf{López Zaragoza Joscelyn Meztli}}\\[0.15cm]
{\large \emph{Directores}}\\[0.5cm]
{\large \textbf{M. en C. Cordero López Martha Rosa}}\\[0.15cm]
{\large \textbf{M. en C. Dorantes González Marco Antonio}}\\[1.25cm]

\hfill {\small Diciembre 2017}
\end{center}
\clearpage

    %Tabla de contenido
    \tableofcontents 
    %Indice de figuras
    \listoffigures
    %Lista de tablas
    \listoftables
    
    \mainmatter

    %Lorem ipsum dolor sit amet, consectetur adipiscing elit. Nunc nec orci ac mauris vestibulum
    %En el cuadro \ref{fig:pato}, se puede ver la imagen de un pato.
    %Como agregar imagenes
    %\begin{figure}[h!]
        %\label{fig:pato}
       % \includegraphics[clip=true,trim=5cm 5cm 5cm 5,width=\linewidth]{cu/meztli}
        %\caption{Imagen de un pato (meztli)}
    %\end{figure}
    
    
    \chapter{Manual de instalacion}
    \section{Presentación}
\begin{itemize}
    \item \textbf{Nombre del sistema:} Sistema de evaluación de la técnica 
    del estilo crol en natación utilizando patrones definidos (SEPACROL)
    \item \textbf{Versión de sistema:} 1.0
    \item \textbf{Tipo de manual:} El presente manual de instalación fue diseñado 
    como una guía para la correcta instalación de las librerías y recursos que el 
    sistema (SEPACROL) requiere para funcionar adecuadamente. Además, implica guiar 
    al usuario en la instalación del sistema en el mismo entorno (PC).
\end{itemize}
Logotipo del sistema
\begin{itemize}
    \item \textbf{Fecha de elaboración:} Octubre, 2017
\end{itemize}
%El sistema \textbf{SEPACROL} queda prohibido cualquier tipo
%de explotación y la reproducción como la distribución o la trasformación
%total o parcial por cualquier medio sin que se tenga el consentimiento
%expresado por escrito por parte de \textbf{SEPACROL}.
\section{Introducción}
\subsection{Objetivo}
El sistema tiene como objetivo ser una herramienta de apoyo para los entrenadores 
de natación, específicamente, para la evaluación de la técnica de crol.

\subsection{Alcance}
El sistema fue diseñado para realizar análisis en nadadores de alto rendimiento de 12 
a 25 años que dominen la técnica de crol. Sin embargo, los principales actores son los 
entrenadores, quienes pueden hacer uso del sistema, siempre y cuando cuenten con los 
elementos requeridos para efectuar el análisis, es decir, un vídeo en vista frontal 
y otro en vista lateral del nadador ejecutando la técnica y que cumplan con los 
estándares técnicos mínimos requeridos.
\section{Descripción de sistema}
\subsection{Descripción del sistema}
Antes del desarrollo del sistema SEPACROL, no se contaba con una herramienta 
especializada para el análisis de los ángulos en la técnica de crol en México. El 
software se postula como una alternativa al análisis subjetivo que sufre la técnica 
deportiva en la natación.\\
El sistema se construyó con base a la biomecánica de la natación, la cual nos indica 
los ángulos ideales en la brazada y patada por fase para una técnica eficiente.
\subsection{Componentes del sistema}
\begin{enumerate}
    \item \textbf{Procesamiento de vídeo:}
    Es el componente que a partir del flujo del video obtiene los frames
    para ello se toma en cuenta la codificación.
    Los frames obtenidos del video se eligen los que cumplen 
    con tres marcas de un nadador.
    \item \textbf{Segmentación de la imagen por color:}
    Se detectan las tres marcas en el brazo del nadador 
    para evaluar su brazada:
    \begin{itemize}
        \item Color verde la del hombro.
        \item Color amarillo la del codo.
        \item Color rojo en la muñeca.
    \end{itemize}
    En la patada se contará con la detección de tres 
    marcas que están colocadas en el siguiente orden:
\begin{itemize}
   \item Color verde en la parte baja de la rodilla.
   \item Color amarillo en el tobillo.
   \item Color rojo recubriendo parte del empeine, recubriendo lateralmente 
   y concluyendo en la parte de la planta del pie, justo por debajo del empeine.
\end{itemize}
    Los colores de cada marca son convertidos en una máscara binaria.
\\*
    \item \textbf{Obtención del cluster (identificación de la marca):}
    En esta etapa del proceso se identifica la marca de color previamente 
    segmentada, procurando minimizar el ruido resultante de la binarización de 
    la imagen.\\
    Se estima que, de acuerdo a la calidad de la segmentación de color, la
    identificación de la marca se pueda hacer mediante agrupación de píxeles 
    semejantes mediante búsqueda en profundidad.
    Finalmente, el cluster resultante es adaptado a una forma regular y se obtiene el
    centroide de dicha figura.
\\*
    \item \textbf{Cálculo de vectores:}
    Por medio del centroide de cada marca, se unen matemáticamente las tres marcas de 
    color: las marcas roja, verde y amarilla formaran un triángulo. Con ello, podrán
    ser calculados dos vectores para poder conformar la fórmula del arco coseno.
    Se debe tomar como origen un centroide para calcular un ángulo:

\begin{longtable}{ | p{5.5cm} | p{5.5cm} | }
	\caption{Marca origen respecto del ángulo a calcular}
	\label{table:marcaorigen}
	\\	\hline
		 \textbf{Marca origen} &\textbf{Ángulo}
		\\ \hline
        \textbf{Verde}&Ángulo de muñeca.
        \\ \hline
        \textbf{Amarilla}&Ángulo hombro.
        \\ \hline
        \textbf{Roja}&Ángulo de codo.
        \\ \hline
\end{longtable}

    \item \textbf{Reconocimiento de patrones:}
    En esta fase, se evaluará que los ángulos obtenidos estén dentro de un rango 
    definido con base en literatura basada en la biomecánica de la natación, 
    así como en los valores obtenidos mediante el análisis de grabaciones 
    realizadas a nadadores de alto rendimiento. Esto, con la finalidad de adaptar 
    la evaluación de patrones a la envergadura y complexión de nadadores mexicanos.
\\*
    \item \textbf{Visualización de resultados:}
\\*
    Elaboración de un reporte de resultados dirigido al entrenador. Dicho reporte 
    contendrá datos tanto de la brazada como de la patada del nadador.
    Contendrá datos tanto del nadador como del entrenador.
\end{enumerate}
\section{Recursos de hardware}
\begin{longtable}{ | p{4.5cm} | p{4.5cm} | p{4.5cm} | }
	\caption{Características de hardware de la PC}
	\label{table:hardwarepc}
	\\	\hline
		 \textbf{Dato} &\textbf{Valor mínimo}&\textbf{Valor recomendado}
		\\ \hline
        \textbf{Procesador}&Intel Core 2 Quad& Intel Core i5
        \\ \hline
        \textbf{Memoria RAM }&2GB&4GB
        \\ \hline
\end{longtable}

\begin{longtable}{ | p{4.25cm} | p{4.25cm} |  p{5.0cm}  | }
	\caption{Características de las cámaras sumergibles}
	\label{table:camaras}
	\\	\hline
		 \textbf{Atributo} &\textbf{Valor mínimo}&\textbf{Valor recomendado}
		\\ \hline
        \textbf{Cámara subacuática}&Tipo Go Pro&Go Pro Hero 3 o superior
		\\ \hline
		\textbf{Resolución}&720p&1080p o superior
		\\ \hline
		\textbf{Cuadros por segundo}&30&60
		\\ \hline
\end{longtable}
\input{manualInstala/cintas}
\section{Recursos de software}
\begin{itemize}
\item \textbf{Matriz de certificación:}
Para el sistema Windows se instalarán las siguientes
librerías. Se recomienda instalarlas desde la 
aplicación \textbf{mini conda}.
\begin{longtable}{ | p{3.0cm} | p{5.5cm} | p{6.0cm} | }
	\caption{Instalación de librerías}
	\label{table:librerias}
	\\	\hline
		 \textbf{Librería} &\textbf{Comando}&\textbf{Uso}
		\\ \hline
        \textbf{Open CV}&conda install -c menpo opencv3&Procesamiento de vídeo como el 
        análisis de las imágenes.
        \\ \hline
        \textbf{Numpy}&conda install numpy&Para el procesamiento de matrices como 
        operaciones matemáticas.
        \\ \hline
        \textbf{Pillow}&conda install pillow&Brinda al intérprete de python la 
        facultad de procesar imágenes.
        \\ \hline
        \textbf{PyGame}&conda install -c cogsci pygame&Conjunto de módulos del 
        lenguaje Python que permiten la creación de videojuegos en dos dimensiones de 
        una manera sencilla. Está orientado al manejo de sprites.
        \\ \hline
        \textbf{Sympy}&conda install sympy&Librería para computación geográfica.
        \\ \hline
        \textbf{Matplotlib}&conda install matplotlib&Creación de graficas  y plots.
         \\ \hline
\end{longtable}
\item \textbf{Restriciones técnicas del sistema}
\begin{longtable}{ | p{6.0cm} | p{5.5cm} | p{5.5cm} | }
	\caption{Restricciones técnicas}
	\label{table:restriccionestec}
	\\	\hline
		 \textbf{Elemento} &\textbf{Descripción}
		\\ \hline
        \textbf{Sistema operativo}&Windows 7, 8, 8.1 o 10
        \\ \hline
        \textbf{Service pack}&SP1 en adelante
        \\ \hline
        \textbf{Sistema gestor de base de datos}&MySQL
        \\ \hline
        \textbf{Intérprete}&Python Shell
        \\ \hline
        \textbf{Tipo de software}&Escritorio o Stand alone
        \\ \hline
\end{longtable}
\end{itemize}
\clearpage
\section{Instalación del sistema}
\begin{itemize}
    \item \textbf{Instalación y configuración del software base}
    \\
    \textbf{Python}
    \\
    Es un lenguaje de programación interpretado, multiparadigma que
    soporta programación orientada a objetos, programación imperativa
    y programación funcional.
    \\
    \item \textbf{Localización}
    \\
    Se localiza en el Disco C (normalmente, la partición principal 
    del disco duro en Windows), en la subcarpeta Archivos de programa.
    \\
    \item \textbf{Procedimiento de Instalación}
    \\
    \begin{itemize}
        \item Entrar al sitio oficial de Python.
        \item Dar click en el que corresponde al sistema operativo Windows.
        \item Descargar la version 3.x con arquitectura x64 o x86 según 
        corresponda la versión de Windows instalada en su PC.
        \item Ejecute el instalador de Python dando doble clic sobre el
        archivo ejecutable.
        \item Dar clic en \textbf{siguiente} hasta llegar al botón \textbf{finalizar}.
        \item Para verificar que está instalado Python, favor de seguir estos pasos:
        \begin{enumerate}
            \item Dar clic en el botón \textbf{inicio}.
            \item Clic en la opción Ejecutar y escribir \textbf{cmd}.
            \item Pulsar Enter
            \item Se abrirá la consola de Windows
            \item Buscar la carpeta de C:\textbackslash{python3.x} y debe mostrar unos 
            cocodrilos y podras dar enter y salir de la consola de python por medio 
            del comando exit().
        \end{enumerate}
    \end{itemize}

        \item \textbf{Procedimiento de configuración}
        \\
        \begin{enumerate}
            \item Modificar las variables de entorno
            \item Modificar el PATH. Para ello, dar doble clic en editar, con ello se 
            situará hasta el cursor hasta el final de la línea. Allí es donde se deberá 
            agregar la ruta \textbf{C:\textbackslash{python3.x}}
            \item Fin de los pasos de instalación.
        \end{enumerate}
\end{itemize}  

\textbf{Open CV}
\\
Librería open source para procesamiento de imágenes. Principalmente utilizada para
computación visual cuenta con estructuras básicas de datos como optimización de matrices
y procesamiento de video.
\textbf{Localización}
\\
Se localiza en el disco C y en los archivos de programa.
\\
\textbf{Procedimiento de Instalación}
\\
\begin{enumerate}
    \item Bajar el Open CV 3 package del sitio oficial. 
    \item Dar click en el que corresponde al sistema operativo Windows.
    \item Instalar en c:/opencv.
    \item Instalar mini conda 
    \item Entrar a la consola de mini conda. 
    \item En la terminal de mini conda escribir: \textbf{conda install opencv3}
    \item Verificar que fue instalado accediendo a python en la terminal, luego, escribir
    \textbf{import cv2} y pulsar Enter, hecho esto, deben aparecer los cocodrilos de python
    en la terminal.
    \item No debe aparecer ningun mensaje de error.
    \item Fin de la instalación.
\end{enumerate}
\textbf{Procedimiento de configuración de OpenCV 3}
\\
Se verifica que este instalado en la siguiente ruta:
\begin{itemize}
    \item C:\textbackslash{Users} \textbackslash{Anaconda}
    \item C:\textbackslash{Users} \textbackslash{Anaconda} \textbackslash{Scripts}
\end{itemize}
De acuerdo con la arquitectura de su PC (x64 o x86), debería ver el directorio \textbf{opencv}
en la ruta que corresponda según la siguiente tabla:
\begin{longtable}{ | p{5.5cm} | p{5.5cm} | p{5.5cm} | }
	\caption{Directorios de OpenCV por arquitectura}
	\label{table:opencvarquitectura}
	\\	\hline
		 \textbf{Arquitectura} &\textbf{Directorio}&\textbf{Ruta}
		\\ \hline
        \textbf{32 bits}&OPENCV\_DIR&C:\textbackslash{opencv} \textbackslash{build} \textbackslash{x86} \textbackslash{vc12}
        \\ \hline
        \textbf{64 bits}&OPENCV\_DIR&C:\textbackslash{opencv} \textbackslash{build} \textbackslash{x64} \textbackslash{vc12}
        \\ \hline
\end{longtable}
\textbf{Parámetros a configurar}
\begin{itemize}
    \item Ninguno.
\end{itemize}   
 
\textbf{MySql}
\\
Es un sistema de gestión de bases de datos relacional desarrollado bajo licencia dual
GPL/Licencia comercial por Oracle Corporation y está considerada como la base datos
open source más popular del mundo.
\\
\textbf{Localización}
\\
Por defecto, se localiza en el Disco C (normalmente, la partición principal del 
disco duro en Windows), en la subcarpeta Archivos de programa.
\textbf{Procedimiento de Instalación}
\\ 
\begin{itemize}
    \item Entrar al sitio oficial de MySql 
    \item Dar click en el que corresponde al sistema operativo Windows.
    \item Descargar la versión mysql-installer-community 5.5 
    \item Abrir el ejecutable.
    \item Hacer doble clic en setup.exe para iniciar el proceso de configuración.
    \item Hacer clic en \textbf{personalizar}.
    \item Hacer clic en \textbf{cambiar}.
    \item En la siguiente ventana en el campo de texto titulado Folder Name, 
    se cambiará la ruta por \textbf{C:\textbackslash{ServerMySQL}} y dar clic en aceptar.
    \item En la siguiente ventana hacer clic en \textbf{siguiente} y MySQL estará 
    listo para instalarse.
    \item Se sugiere saltar el registro y dar \textbf{siguiente}.
    \item La instalación estará lista cuando aparezca la ventana MySQL Sing-Up.
\end{itemize}
\textbf{Procedimiento de configuración}
\begin{itemize}
   \item Para iniciar la configuración dejar marcado MySQL 
   Server Now y dar clic en \textbf{finalizar}.
   \item Dar clic en Server Machine dejar el puerto por defecto (3306).
   \item Elegir la configuración estándar y dar clic en siguiente.
   \item Verificar que están marcados:
   \begin{itemize}
       \item Install As Windows Service.
       \item Launch the MySQL Server Automatically.
   \end{itemize}
    \item Dar clic en siguiente.
    \item Crear una contraseña de administrador como se solicita en la ventana.
    Verificar que root este marcado como \textbf{enable root access from remote 
    machine} y dar clic en siguiente.
    \item Dar clic en ejecutar para iniciar el servidor de MySQL y dar clic en finalizar.
    \item Entrar a una consola de MySQL y escribir el siguiente comando: mysql -u root -p
    y presione Enter; acto seguido, se le pedirá que ingrese la contraseña de administrador
    que configuró en pasos anteriores.
    \item No debe salir ningun error.
    \item Se termino el proceso de configuración.
\end{itemize}
\textbf{SQL Alchemy}
\\
Es un Object Relational Mapper, es presentado y descrito completamente para python. La 
persistencia es automática.
\textbf{Localización}
\\
Por defecto, se localiza en el Disco C (normalmente, la partición principal del 
disco duro en Windows), en la subcarpeta Archivos de programa.
\\
\textbf{Procedimiento de Instalación}
Los pasos son los siguientes:
\begin{itemize}
    \item Ir al sitio oficial de SQL Alchemy y descargar el ejecutable según la 
    de su PC, es decir, 32 bits o 64 bits.
    \item Entrar a la consola y entrar a carpeta \textbf{C:\textbackslash{Python3.x}\textbackslash{python.exe}}
    \textbackslash{setup.py} install
    \item Instalación finalizada
\end{itemize}
\textbf{Procedimiento de configuración}
\begin{itemize}
    \item Ninguno
\end{itemize}
\textbf{Pyramid}
\\
Es un framework para construir aplicaciones web desarrolladas con python.
\\
\textbf{Localización}
\\
Por defecto, se localiza en el Disco C (normalmente, la partición principal del 
disco duro en Windows), en la subcarpeta Archivos de programa.
\\
\textbf{Procedimiento de Instalación}
\\
Los pasos son los siguientes:
\begin{itemize}
    \item Ir al sitio oficial de Pyramid y descargar el ejecutable según la 
    de su PC, es decir, 32 bits o 64 bits.
    \item Entrar a la consola de Windows e instalar con los siguientes comandos:
    \begin{enumerate}
        \item pip install pyramid==1.9.1
    \end{enumerate}
    \item La instalación habrá finalizado cuando no se envíe ningun mensaje de error.
\end{itemize}

\textbf{Procedimiento de configuración}
\begin{itemize}
    \item Ninguno.
\end{itemize}
\clearpage
\section{Glosario}
\textbf{Glosario}

\begin{longtable}{ | p{2.5cm} | p{4.5cm} |  }
	\caption{Glosario del manual de instalación}
	\label{table:glosariomanuinsta}
	\\	\hline
		 \textbf{Término} &\textbf{Descripción}
        \\ \hline
        \textbf{Aplicación de escritorio}&Es aquella que se encuentra instalada en 
        el disco duro de la computadora y no requiere de conexión a internet para 
        que el usuario pueda hacer uso de sus funciones.
        \\ \hline
        \textbf{Consola} &Es el programa informático que provee una interfaz de 
        usuario para acceder a los servicios del sistema operativo.
        \\ \hline
        \textbf{Lenguaje de programación}&Es un lenguaje formal diseñado para 
        realizar procesos que pueden ser llevados a cabo por máquinas como las 
        computadoras. Estos pueden usarse para crear programas que controlen el 
        comportamiento físico y lógico de una máquina, para expresar algoritmos 
        con precisión, o como modo de comunicación humano.
        \\ \hline
        \textbf{Framework}&Es un conjunto estandarizado de conceptos, prácticas y 
        criterios para enfocar un tipo de problemática particular, que sirve como 
        referencia para enfrentar y resolver nuevos problemas de índole similar.
        \\ \hline
\end{longtable}
%Falta cluster, open source, terminal/consola, ejecutable, interfaz, servidor
\clearpage
\section{Sitios oficiales del software}
\textbf{Páginas de sitios oficiales}
\begin{longtable}{ | p{2.5cm} | p{4.5cm} |  }
	\caption{Páginas de sitios oficiales}
	\label{table:sitiosoficiales}
	\\	\hline
		 \textbf{Software} &\textbf{URL}
		\\ \hline
        \textbf{Python}&https://www.python.org
        \\ \hline
        \textbf{Open CV}&https://opencv.org
        \\ \hline
        \textbf{MySQL}&https://www.mysql.com
        \\ \hline
        \textbf{SQL Alchemy}& http://docs.sqlalchemy.org
        \\ \hline
        \textbf{Pyramid}& https://trypyramid.com/
        \\ \hline
\end{longtable}
\clearpage
 
    
    %Dev_MA: COMENTADO por no existencia de imagen
    %\begin{figure}[h!]
    %    \label{fig:CU5.0}
    %   \includegraphics[clip=true,trim=5cm 5cm 5cm 5,width=\linewidth]{cu/CargarVideo}
    %	\caption{Imagen de CUCG5.0 Cargar Vídeo}
    %\end{figure}
    
    
    
    %\backmatter
    %\bibliography{referencias}
    %\bibliographystyle{unsrt}
    %\printglossary[title=Glosario]
    \end{document}
    