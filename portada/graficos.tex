
\thispagestyle{empty}
\begin{minipage}[c][0.15\textheight][c]{0.2\textwidth}
\begin{center}
    \includegraphics[height=3.5cm]{marcoTeorico/imagenes/general/logo_ipn}
\end{center}
\end{minipage}
\begin{minipage}[c][0.1\textheight][c]{0.65\textwidth}
\begin{center}
    {\LARGE Instituto Politécnico Nacional}
    \vspace{.3cm}
    \hrule height2.5pt
    \vspace{.1cm}
    \hrule height1pt
    \vspace{.5cm}
    {\Large Escuela Superior de Cómputo}
    \\
    \vspace{.4cm}
    {\Large ESCOM}
\end{center}
\end{minipage}

\begin{minipage}[c][0.65\textheight][t]{0.2\textwidth}
\begin{center}
\hskip2pt
\vrule width2.5pt height17cm
        \hskip1mm
        \vrule width1pt height17cm \\
        \includegraphics[height=3cm]{marcoTeorico/imagenes/general/logo_escom}
        \end{center}

\end{minipage}
\begin{minipage}[c][0.6\textheight][t]{0.65\textwidth}
  \begin{center}
  	{\large \textit{Trabajo Terminal}}
    \\
     \vspace{.01cm}
    \end{center}
    {\large{ ``Sistema de evaluación de la técnica del estilo crol en natación utilizando patrones de distancia (SEPACROL)''} }%%Titulo
    \\
    \vspace{.01cm}
	\begin{center}
    {\large \textit{Trabajo Terminal No. 2016-B006}} %%Número de trabajo terminal
    \\
    \vspace{.5cm}
    {\large \textit{Presentan}}

    \begin{center} %%Alumnos
		 \textbf{Hoyos Estrada Edgar Omar} \linebreak
        \textbf{López Zaragoza Joscelyn Meztli} \linebreak
       
    \end{center}

    \vspace{0.3cm}

   {\large \textit{Director}}
   \\
    \textbf{{M. en C. Marco Antonio Dorantes González}}\\*
    \textbf{{M. en C. Martha Rosa Cordero López}}
    \vspace{0.5cm}
	 \\
    {\textbf{RESUMEN}}
    \end{center}
Se plantea crear un sistema para el análisis de la técnica de natación denominada crol. Dicho sistema estará diseñado como una 
    herramienta de apoyo para entrenadores que únicamente tendrán que proporcionar al sistema un vídeo del
    nadador ejecutando la técnica; posteriormente, el sistema realizará un análisis sobre cada cuadro del vídeo aplicando técnicas
    de filtrado, segmentación, extracción de características y esqueletización. Finalmente, de la información obtenida en el análisis,
    el sistema realizará el diagnóstico de la técnica del nadador por medio de reconocimiento de patrones de distancia.
	 

\vspace{0.2cm}
\textbf{Palabras clave}: Sistema, Patrón, Evaluación, Natación, Vídeo.

\vspace{0.5cm}
\raggedleft{Mayo, 2017}

\end{minipage}